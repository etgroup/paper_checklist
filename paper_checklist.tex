% ----------------------------------------------------------------
% Article Class (This is a LaTeX2e document)  ********************
% ----------------------------------------------------------------

\documentclass{ctexart}

\usepackage{latexsym,bm}
\usepackage{amssymb}
\usepackage{amsfonts}
\usepackage{graphicx}

\usepackage{setspace}
\usepackage{supertabular}
\usepackage{indentfirst}
\usepackage{multirow}
\usepackage{ulem}
\usepackage{fontspec}

\usepackage{longtable}



\usepackage[unicode]{hyperref}
\usepackage{xcolor}
\usepackage{cite}



\usepackage{eso-pic}
\usepackage{tikz}

\usetikzlibrary{shapes.geometric, arrows}

% ----------------------------------------------------------------
\begin{document}

\begin{center}\label{tab:8}
\begin{longtable}{|l|l|p{0.8\textwidth}|l|}\caption{论文检查表}\\

\hline \multicolumn{1}{|c|}{} & \multicolumn{1}{c|}{\textbf{编号}} & \multicolumn{1}{p{0.8\textwidth}|}{\textbf{项目}}& \multicolumn{1}{c|}{\textbf{备注}} \\ \hline
\endfirsthead

\multicolumn{4}{c}%
{{\bfseries \tablename\ \thetable{}论文检查表续表}} \\
\hline \multicolumn{1}{|c|}{} &
\multicolumn{1}{c|}{\textbf{编号}} &
\multicolumn{1}{p{0.8\textwidth}|}{\textbf{项目}} &
\multicolumn{1}{c|}{\textbf{备注}} \\ \hline
\endhead

\hline \multicolumn{4}{|r|}{{下页继续}} \\ \hline
\endfoot

\hline \hline
\endlastfoot

 & 1 & \textcolor[rgb]{0.00,0.00,1.00}{\textbf{Language}}& \\\hline
$\bigcirc$& 1.1 & 不用第一/第二人称:I/we/you 以 one替,或改用被动语态 & $\dagger$\footnote{$\dagger$适用英文期刊及会议论文,$\ddagger$适用学位论文}\\\hline
$\bigcirc$& 1.2 & 不用通俗用语,使用专业词汇& $\dagger$\\\hline
$\bigcirc$& 1.3 & 不用简略形式:can't, it's, that's,用:cannot, it is, that is & $\dagger$\\\hline
$\bigcirc$& 1.4 & 不用多重从句:一句不能出现多个that /what /which /where...& $\dagger$\\\hline
$\bigcirc$& 1.5 & 不用武断性的评语,如absolutely, none, never等词语 & $\dagger$\\\hline
$\bigcirc$& 1.6 & 注意拉丁词应斜体:\it{et al、etc、in situ、in vivo}...& $\dagger$\\\hline
$\bigcirc$& 1.7 & 正文中不用``="代替``is"/``equals";不用``+"代替``and"& $\dagger$\\\hline
$\bigcirc$& 1.8 & 正文中公式采用$f(x)=x/a$,而不采用$f(x)=\frac{x}{a}$ 的形式& $\dagger$\\\hline
$\bigcirc$& 1.9 & 注意数字(正体)与单位(斜体)之间有空格:$0.2~mm$,非$0.2mm$,\LaTeX 代码为\verb+$0.2~mm$+& $\dagger, \ddagger$\\\hline
$\bigcirc$& 1.10 & 注意温度/角度符号与数字之间无空格:$36.7^\circ$,\LaTeX 代码为\verb+$36.7^\circ$+ & $\dagger, \ddagger$\\\hline
$\bigcirc$& 1.11 & 注意英文标点之后均有空格:a, b 而非  a,b & $\dagger, \ddagger$\\\hline
$\bigcirc$& 1.12 & 注意公式格式应统一,不论是在文中或公式中: 矩阵 $\mathbf{A}$, 矢量$\mathbf{v}$,函数$f(x)$& $\dagger, \ddagger$\\\hline
$\bigcirc$& 1.13 & 注意一些拉丁字母:$\mu m$非$um$,$j\omega$非$jw$ & $\dagger, \ddagger$\\\hline
$\bigcirc$& 1.14 & 正文交叉引用时,使用红/蓝色超链接 & $\dagger, \ddagger$\\\hline%,如:Table. \ref{tab:8}, 文献[8]
$\bigcirc$& 1.15 & 注意段落不要太长,以3到5个句子为宜;不用数字作为句子的开头 & $\dagger, \ddagger$\\\hline
$\bigcirc$& 1.16 & 当需要连续介绍/描述多于三种事物/符号时,应以: 
\begin{itemize}
  \item ECT:……
  \item ERT:……
  \item EMT:……
\end{itemize}
& $\dagger, \ddagger$\\\hline
$\bigcirc$& 1.17 & 当需要按一定顺序介绍/描述流程时,应 辅以序号:
\begin{enumerate}
  \item 系统标定,……
  \item 进行测量,……
  \item 结束测量,……
\end{enumerate}
& $\dagger, \ddagger$\\\hline
$\bigcirc$& 1.18 & 注意冠词的使用,多数情况下名词前都需要有冠词a/an/the & $\dagger$\\\hline
$\bigcirc$& 1.19 & 注意在“the number of channel”中,channel不用冠词 & $\dagger$\\\hline
$\bigcirc$& 1.20 & 在英文论文中,专业词汇不需要首字母大写,仅对缩写大写,如:process tomography (PT);如果该词出现频率很低,则不需要提供其缩写。 & $\dagger$\\\hline
$\bigcirc$& 1.21 & 在中文论文中,专业词汇先中文(首字母大写的英文全称,英文缩写) ,如:过程成像(Process Tomography,PT)& $\ddagger$\\\hline
$\bigcirc$& 1.22 & \textbf{当且仅当}专业词汇在首次出现时,对进行缩写/解释 & $\dagger, \ddagger$\\\hline
$\bigcirc$& 1.23 &  & $\dagger$\\\hline
\hline

 & 2 & \textcolor[rgb]{0.00,0.00,1.00}{\textbf{Title}}& $\dagger$\\\hline
$\bigcirc$ & 2.1 & 单词个数小于$15$& $\dagger$\\\hline
$\bigcirc$ & 2.2 & 不包含结果或结论性的内容 & $\dagger$\\\hline
$\bigcirc$ & 2.3 & 各单词首字母大写& $\dagger$\\\hline
$\bigcirc$ & 2.4 & 冠词、介词、连词和'to'等虚词不需要首字母大写& $\dagger$\\\hline
$\bigcirc$ & 2.5 & 不出现Study/Investigation/Research等& $\dagger$\\\hline
$\bigcirc$ & 2.5 & 不出现小学科领域内的简略词,如ECT/ERT& $\dagger$\\\hline
$\bigcirc$ & 2.5 & 不出现new, novel等& $\dagger$\\\hline
\hline


$ $ & 3 & \textcolor[rgb]{0.00,0.00,1.00}{\textbf{Abstract}}& $\dagger$\\\hline
$\bigcirc$ & 3.1 & 摘要不出现图、表、公式& $\dagger$\\\hline
$\bigcirc$ & 3.2 & 摘要不能照抄正文(引言、结论)& $\dagger$\\\hline
$\bigcirc$ & 3.3 & 注意涵盖:目的/问题、方法、结果/发现、结论/影响& $\dagger$\\\hline
$\bigcirc$ & 3.4 & 摘要只出现1$\sim$2次:“In this paper, ...”& $\dagger$\\\hline
\hline

 & 4 & \textcolor[rgb]{0.00,0.00,1.00}{\textbf{Introduction}}& $\dagger$\\\hline
$\bigcirc$& 4.1 & 每一段都有清晰的主题/topic:
\begin{itemize}
  \item 第一段:介绍研究背景、应用领域等
  \item 第二段:介绍他人的方法,评价贡献,并引出问题
  \item 第三段:介绍自己的思路/方法,所做贡献
\end{itemize}
 & $\dagger$\\\hline
$\bigcirc$& 4.2 & 语态使用正确,对应上述各段: 
\begin{itemize}
  \item 一般现在时:general background and description of the work
  \item 现在完成时:past to present solutions
  \item 一般过去时:my contribution,也可用一般现在时
\end{itemize}

& $\dagger$\\\hline
$\bigcirc$& 4.3 & 注意减少使用次数:“In this paper, ...” & $\dagger$\\\hline
\hline
 & 5 & \textcolor[rgb]{0.00,0.00,1.00}{\textbf{Method}}& $\dagger$\\\hline
$\bigcirc$& 5.1 & 完整、简洁 & $\dagger$\\\hline
$\bigcirc$& 5.2 & 步骤/方法清晰、合理 & $\dagger$\\\hline
$\bigcirc$& 5.3 & 至少包括一幅图,用于说明方法 & $\dagger$\\\hline
$\bigcirc$& 5.4 & 注意多使用数学公式说明问题 & $\dagger$\\\hline
$\bigcirc$& 5.5 &  & $\dagger$\\\hline
\hline

 & 6 & \textcolor[rgb]{0.00,0.00,1.00}{\textbf{Results, Discussions and Conclusions}}& $\dagger$\\\hline
$\bigcirc$& 6.1 & 数据/结果,尽量以图的形式呈现 & $\dagger$\\\hline
$\bigcirc$& 6.2 & Being convincing and credible & $\dagger$\\\hline
$\bigcirc$& 6.3 & Interpret the results without repeating them & $\dagger$\\\hline
$\bigcirc$& 6.4 & 结论:清晰不含糊,让人印象深刻 & $\dagger$\\\hline
$\bigcirc$& 6.5 & 结论不重复摘要和引言的内容 & $\dagger$\\\hline
$\bigcirc$& 6.6 & 避免冗长而无实质内容 & $\dagger$\\\hline
$\bigcirc$& 6.7 & 注意:Results, Discussions, Conclusions都是复数 & $\dagger$\\\hline
\hline

 & 7 & \textcolor[rgb]{0.00,0.00,1.00}{\textbf{Acknowledgments}}& $\dagger$\\\hline
$\bigcirc$& 7.1 & 包括最新的基金号 & $\dagger$\\\hline
$\bigcirc$& 7.2 & 感谢对本论文写作有帮助的人或机构,必须写明for... & $\dagger$\\\hline
$\bigcirc$& 7.3 & 1、学位论文内的详细标注,用于学位论文内“参加科研情况说明”:

(1)	国家自然科学基金面上项目,项目批准号:61671319,项目名称:基于ECT/EMT双模成像的气/液/固三相流过程参数表征方法

(2)	国家自然科学基金国家重大科研仪器研制项目,项目批准号:61627803,项目名称:基于电/磁双模层析成像的高固含率气液固三相流态化实验装置

2、学位论文内的简洁标注,放置于学位论文标题页右上角,右侧对齐:

国家自然科学基金面上项目:61671319

国家重大科研仪器研制项目:61627803 & $\ddagger$\\\hline
$\bigcirc$& 7.4 & 感谢有帮助的人或机构,必须写明因何感谢,for... & $\dagger$\\\hline
$\bigcirc$& 7.5 & This research is financially supported by NSFC (Nos. 61671319 and 61627803).  & $\dagger$\\\hline
$\bigcirc$& 7.5 & The authors would like to thank the financial support from NSFC (Nos. 61671319 and 61627803).  & $\dagger$\\\hline
\hline

& 8 & \textcolor[rgb]{0.00,0.00,1.00}{\textbf{References}}& $\dagger$\\\hline
$\bigcirc$& 8.1 & 使用文献软件(EndNote/Jabref)或Baidu/Google学术 & $\dagger$\\\hline
$\bigcirc$& 8.2 & 文献引用时,先正面和客观的评价,再说问题& $\dagger$\\\hline
$\bigcirc$& 8.3 & 参考文献条目中,提供DOI链接& $\dagger$\\\hline
$\bigcirc$& 8.4 & 引用必要的经典文献,$3\sim6$篇& $\dagger$\\\hline
$\bigcirc$& 8.5 & 引用最近5年的文献,占比$50\%$以上& $\dagger$\\\hline
$\bigcirc$& 8.6 & 引用所投期刊的文献,5篇以上& $\dagger$\\\hline
$\bigcirc$& 8.7 & 引用尽可能列出所有作者,但第三及之后的作者用\it{et al} 代替& $\dagger$\\\hline
$\bigcirc$& 8.8 & 一般不引用教科书/书/学位论文& $\dagger$\\\hline
$\bigcirc$& 8.9 & 可以适当引用所投期刊AE的论文& $\dagger$\\\hline
$\bigcirc$& 8.10 & 一般参考文献引用编号须放在句末 & $\ddagger$\\\hline
$\bigcirc$& 8.11 & 参考文献交叉引用的编号不能出现在节标题中& $\ddagger$\\\hline
$\bigcirc$& 8.12 & 对姓名(特别是中国人),不能缩写姓,可缩写名,如:Ziqiang Cui; Z. Cui; Z. Q. Cui; Cui, Ziqiang; Cui, Z 等。绝对不能是:Ziqiang C等方式 & $\ddagger$\\\hline
$\bigcirc$& 8.13 & & $\ddagger$\\\hline
$\bigcirc$& 8.14 & & $\ddagger$\\\hline
$\bigcirc$& 8.15 & & $\ddagger$\\\hline
\hline

 & 9 & \textcolor[rgb]{0.00,0.00,1.00}{\textbf{Figures and Tables}}& $\dagger$\\\hline
$\bigcirc$& 9.1 & 所有数据都以excel表格保存,并注明对应的图/表 & $\dagger$\\\hline
$\bigcirc$& 9.2 & 图片分辨率$>600~dpi$ & $\dagger$\\\hline
$\bigcirc$& 9.3 & 图片格式为tiff/eps/pdf/jpg & $\dagger$\\\hline
$\bigcirc$& 9.4 & 图片宽度:半幅7.5 cm,全幅15 cm& $\dagger$\\\hline
$\bigcirc$& 9.5 & 使用MATLAB软件的图片导出功能,而不是截屏& $\dagger$\\\hline
$\bigcirc$& 9.6 & 所有图片按照文中的顺序命名(fig\_8.tiff),并集中保存& $\dagger$\\\hline
$\bigcirc$& 9.7 & 横纵坐标轴有名称和单位& $\dagger$\\\hline
$\bigcirc$& 9.8 & 图表字体符合要求,默认用Times New Roman/Arial& $\dagger$\\\hline
$\bigcirc$& 9.9 & 同一类型的文字,字号保持一致,且$>8~pt$& $\dagger$\\\hline
$\bigcirc$& 9.10 & 准确、清楚、有条理的图片标记,插图上所有元素对位整齐 & $\dagger$\\\hline
$\bigcirc$& 9.11 & 内容应占据整图$90\%$以上空间,不能留白太多& $\dagger$\\\hline
\hline

 & 10 & \textcolor[rgb]{0.00,0.00,1.00}{\textbf{Others}}& $\dagger$\\\hline
$\bigcirc$& 10.1 & 在天大学位论文格式规定中,要求使用“第1章”,而非第一章。 & $\ddagger$\\\hline
\end{longtable}

\end{center}




\end{document}
% ----------------------------------------------------------------
